\section{Introduction}
\label{S:introduction}

\begin{itemize}
  \item \emph{What is this research plan about?}
  \item \emph{What are the (high-level) research gaps?}
  \item \emph{What is the overall goal?}
\end{itemize}

\hrulefill

\begin{itemize}
 
 \item Motivation: (scientific) importance of environmental monitoring spatio-temporal phenomena 
\cite{Detweiler2015, Hitz2015}
 \item Example applications: exploring oceans, studying volcanoes, surveying agricultural fields
\unsure{How application-specific should this be? e.g. monitoring fields vs monitoring in general} 
\unsure{Should I focus on 
one application for the plan?}
 \item Large-scale environmental monitoring is becoming more relevant: focus on saving resources and 
understanding natural disasters/climate change and its implications

 \item Advantages of using autonomous systems over manual procedures \cite{Detweiler2015}: increase data 
collection efficiency, can be cheaper, more flexible
 
 \item Current research gaps ($\rightarrow$ connect to related work sections)
 \begin{itemize}
  \item Planning paths for efficient data-gathering given constraints (vehicle mobility, time, fuel)
  \item Adaptive sampling
  \item Operational issues, cooperative planning?
 \end{itemize}

 \item The proposed research aims at providing (IPP) solutions for environmental monitoring using aerial 
robots \unsure{Be more specific.}
 \item Section \ref{S:related_work} of this research plan overviews related work in the field. The proposed 
approach, and its scientific contributions, are detailed in Section \ref{S:proposed_approach}. Finally, 
Section \ref{S:time_schedule} outlines the planned time schedule of the thesis.
 
\end{itemize}