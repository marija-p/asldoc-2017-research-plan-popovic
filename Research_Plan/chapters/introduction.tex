\section{Introduction}
\label{S:introduction}

Autonomous mobile robots allow us to interact with our world in an unprecedented manner. Equipped 
with modern sensors, such devices can collect valuable data in many practical scenarios, helping us better 
understand surrounding environments and their evolution. Owing to recent advances in hardware, artificial 
intelligence, and computational capabilities, information-gathering technologies are becoming more 
common in a wide variety of applications, ranging from indoor spatial mapping~\cite{Charrow2015, 
Heng2014} to outdoor environmental monitoring~\cite{Dunbabin2012, Detweiler2015}.

Small rotary-wing micro aerial vehicles (MAVs), in particular, offer a timely and cost-effective method of 
collecting high-resolution data in sensing applications. A combination of their flexibility, maneuverability, 
and ability to fly at low altitudes~\cite{Anthony2014} permits more detailed inspections of dangerous or 
hard-to-reach regions with greater coverage compared to conventional approaches generally based on manual 
sampling or static sensor networks~\cite{Dunbabin2012, Krause2008, Phillips2012}. As a motivating example, 
consider the problem of monitoring agricultural crops on a field. Here, it is necessary to model the 
evolution of different interrelated physical variables~\cite{Cardina1997, Anthony2014, Detweiler2015} to 
assess crop health and develop integrated workflows supporting sustainability and economic gain. 
Unfortunately, traditional sampling campaigns often involve low-resolution aerial imagery~\cite{DeCastro2013} 
or on-ground methods~\cite{Borra-Serrano2015}, which are expensive and inefficient. In agriculture and other 
% industries, such as marine biology~\cite{Hitz2015, Singh2010, Jadidi2016}, structural 
inspection~\cite{Isler2016,Bircher2016,Heng2015}, and search and rescue~\cite{Singh2009,Lim2015}, aerial 
robots 
emerge as a flexible, cost-efficient alternative for obtaining accurate maps with large spatio-temporal 
coverage. Similar capabilities are also an important prerequisite for autonomous robots such as 
self-driving cars that navigate in unstructured environments.

However, mapping complex environments with mobile robots is a challenging problem. Often, 
quantifying target processes requires managing a large amount of spatially and temporally diverse 
data~\cite{Dunbabin2012}. Given a target environment, a robot can obtain sensor measurements to gain 
information. In doing so, however, it incurs mobility cost in terms of fuel, energy, or time, which preclude 
it from gathering data at arbitrary positions. Moreover, on-board sensing devices are often hard to recharge, 
repair, or replace, posing limitations to their use \cite{Singh2009}. The problem faced thereby involves 
trading off information gain and resource consumption efficiently to generate information-rich trajectories. 
In doing so, the robot must also reconcile exploitation and exploration by deciding whether to improve local 
parts of the map or visit unknown regions~\cite{Charrow2015a}.

There is extensive prior work concerning path planning for exploration and 
mapping using mobile robots. Nonetheless, algorithms to date have not fully leveraged the fidelity and speed 
offered by modern sensing devices. One pressing issue is developing efficient representations to capture 
large-scale spatio-temporal processes suitable for motion planning with informative measures. Another 
challenge is scalability~\cite{Hollinger2014}, as evaluating candidate measurement locations for 
high-dimensional, continuous configuration spaces can incur significant computational cost. In active 
applications, a further important element is adaptivity, whereby new sensor measurements are conditioned 
on the outcomes of earlier ones. This requirement complicates the decision-making process~\cite{Lim2015}. 
Finally, an under-explored practical aspect is the portability of path planning algorithms 
between different domains with similar set-ups, as most developments thus far have been 
application-specific~\cite{Dunbabin2012}.

The proposed thesis aims at providing solutions for informative path planning by addressing the research 
gaps above. Within its scope, the key development will be a scalable, flexible framework for robotic 
navigation in mapping applications. The expected scientific contributions will be within individual 
modules of the planning pipeline. These are:

\begin{itemize}
 \item A mapping framework that incorporates sensor data necessary for motion-based informative 
planning, accounts for measurement uncertainty, and is extendable to unknown 3D environments,
 \item An adaptive, informative planning method based on the environmental model which respects constraints 
on the budget and device dynamics, and
 \item The consideration of application-specific informative objectives within the planning framework.
\end{itemize}

The developed methods will be evaluated in two main scenarios: agricultural monitoring and industrial 
inspection. However, they will be formalized in a generic manner such that they can be used in other 
mapping applications.

Section \ref{S:related_work} of this research plan overviews related work in the field. The proposed 
approach is detailed in Section \ref{S:proposed_approach}. Finally, 
Section \ref{S:time_schedule} outlines the planned time schedule of the thesis.