\section{Proposed Approach}
\label{S:proposed_approach}

The proposed research will focus on developing an IPP framework for a mobile agent 
performing active exploration and mapping. This framework will be formulated in a 
generic manner, based on the decomposition in Section~\ref{S:related_work}, to allow for portability between 
different applications. In this thesis, the developments will be driven by and evaluated for a MAV in two 
representative use-cases: (3.1) Agricultural monitoring and (3.2) Indoor exploration.  As outlined in the 
following sub-sections, the progress for each set-up will be split into stages based on its associated work 
packages.

\subsection{Agricultural Monitoring}

Agricultural monitoring was chosen as a environmental use-case for IPP that 
has gauged significant interest over the past few years~\cite{Detweiler2015, Cardina1997, Anthony2014}. 
In particular, the proposed research will focus on the problem of active weed classification using a 
MAV: a subject of several recent related works~\cite{Vivaldini2016, Sadat2015}. The MAV is 
equipped with a specialized weed classifier unit supplying sensor measurements. Generally, this task can be 
formulated as a 2.5D problem involving the 
search for most informative measurement locations in a 3D configuration space over a 2D environmental model 
(farmland). Here, a key trade-off arises because the same point can be observed from different altitudes; thus 
the planning unit must account for degrading sensor accuracy with increased altitude and coverage. Moreover, 
it must given limited battery and computational capacities.

\subsubsection{Phase 1}

The first phase of the research will aim to establish an efficient problem set-up and procure 
preliminary results. Initially, the weed classifier will be a heuristically tuned probabilistic sensor model 
providing uncorrelated point measurements and reflecting the altitude dependency discussed above. The main 
contribution will be a viewpoint-based planning algorithm that accounts for this trade-off and generates 
dynamically feasible paths for the MAV. The developed framework will be modular; as such, different 
informative objectives and planning methods will be evaluated. Experiments will be conducted in simulation and 
on a real platform using artificial weed distributions.

\subsubsection{Phase 2}

In the second phase, the project will focus on integrating the IPP framework with a real classifier sensor 
model 
and refining its elements accordingly. In particular, the 
environmental model will be improved to capture agricultural expert knowledge involving correlations in 
weed distributions and accept prior data~\cite{Pena2013}. As well as demonstrating the flexibility of the 
proposed framework, this phase will be an important step towards practical MAV applications in agriculture.

\subsubsection{Phase 3}

The final phase of the project will be oriented towards developing a fully integrated system for field 
experiments with real weeds and outdoor state estimation. A local obstacle collision avoidance module will 
also be incorporated. The framework will be evaluated thoroughly with 
respect to various planning objectives and approaches, as well as compared to state-of-the-art approaches.

As outlined in Section~\ref{S:time_schedule}, the developments for this application will be in line with Work 
Package 3: UAV Navigation for the Horizon 2020 project, Flourish.

\subsection{Indoor exploration}

The second use-case will be a generic indoor exploration problem for a MAV relying on vision-based 
sensing, which covers a range of inspection and reconstruction scenarios, including those arising in mining, 
construction, and search and rescue. The aim is to solve the problem of exploring an unknown 3D space by 
autonomously constructing a map as efficiently and quickly as possible. This task has been tackled in several 
recent works~\cite{Charrow2015a, Bircher2016, Heng2015}, widely differing in their approaches. In addition to 
planning with constraints on battery and computational capacities, open questions remain in defining suitable 
objective functions over an efficient volumetric representation of space and planning safe, obstacle-free 
paths based on this model. All following developments will be thoroughly evaluated in experimental set-ups.

\subsubsection{Phase 1}

The first phase of the research will aim to establish a scalable environmental model for planning by 
extending the approaches from the agricultural monitoring use-case to 3D. The planning algorithm will 
combine elements of global viewpoint-based planning and local optimization to cater for the dynamic 
feasibility of the MAV. Initially, a primitive collision avoidance module will be used based on simplifying 
assumptions of unknown space. Different informative objectives and planning methods will be investigated.

\subsubsection{Phase 2}

The second phase will focus on extending the environmental model to provide a foundation for more 
complex decision-making. In particular, the relationship between global and local 
planning for safe operation with respect to the map representation will be considered. To achieve this, 
aspects of multi-objective optimization will be integrated into the framework. A thorough analysis of new 
features and their effects will be carried out.

\subsubsection{Phase 3}

In the final phase, the fully integrated system will be evaluated against benchmarks with a view 
towards the fidelity of volumetric reconstructions obtained. Time-permitting, this phase will investigate the 
possibility of incorporating elements of active SLAM to account for MAV state uncertainty. Combining the work 
from the two use-cases, the IPP framework will be fully documented and released open-source. 
Possible future research directions will also be thoroughly discussed.