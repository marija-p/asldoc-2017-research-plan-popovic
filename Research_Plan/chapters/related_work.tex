\section{Related Work}
\label{S:related_work}

\begin{itemize}
	\item \emph{Which major works consider a similar context?}
	\item \emph{Which works are addressing same/similar problem and why are these works insufficient (Gaps in state of the art)?}
	\item \emph{Which works use a similar methodology?}
\end{itemize}

\hrulefill

\remember{Approx. 40 references}

To discuss:

\subsection{UAVs for environmental monitoring}

\begin{itemize}
 
 \item General overview of application areas \cite{Detweiler2015}: marine biology \cite{Hitz2015, Singh2010}, 
gas detection \cite{Marchant2014}, signal mapping \cite{Jadidi2016}, geophysics \cite{Muscato2012}, etc.
 
 \item Focus on agriculture \cite{Vivaldini2016, Anthony2014, Krienke2015} \unsure{Separate sub-section for 
agriculture?}
 \begin{itemize}
   \item Example: developing target-specific weed management methods can improve yield, leading to 
sustainability and economic gain \cite{Cardina1997}
 \end{itemize}
 
 \item There have been significant advancements. Emerging research trends \cite{Detweiler2015}: cooperative 
robotic teams, robotic and wireless sensor network interaction, adaptive sampling, model-aided informative 
planning
 \item Open challenges: Real-time operation in hostile/large/dynamically varying environments, handling 
complex maps and data, safety, cooperative robotic teams
 \item Aim: push the boundaries of robot science and advance knowledge of the environment and its processes

\end{itemize}

\unsure{How generic should the following sub-sections be?}

\subsection{Informative path planning}
\begin{itemize}

 \item In autonomous environmental monitoring, a key challenge is planning paths to track evolving 
processes while respecting sensing and mobility limitations. Formally, this problem is known as informative 
path planning (IPP) \cite{Singh2009} and has been studied extensively in the context of robotics and related 
fields. Unlike in distance-based planning \cite{Dijkstra1959}, where the goal is to find a shortest 
route between two locations, the paths in IPP are subject to a budget constraint which limits the number of 
informative measurements that can be taken.

 \item Connect to static sensor placement problem \cite{Krause2008}, where the aim is to find a fixed set of 
locations for sensors to observe a given phenomenon

 \item 3 ingredients for model-aided path planning \cite{Detweiler2015} \unsure{Am I researching one aspect 
in 
detail?} \remember{Where are my contributions?}
   \begin{itemize}
     \item Environmental model: Gaussian Processes \cite{Rasmussen2006, Hitz2015, Binney2013} occupancy grids 
\cite{Elfes1989}
     \item Objective function: describe submodularity \cite{Krause2011}, discuss entropy-based vs. 
alternative \cite{Girdhar2015} formulations
     \item Planning algorithm: continuous \cite{Hitz2015, Hollinger2014, Marchant2014} vs. discrete 
\cite{Binney2013}, adaptive \cite{Hitz2015} vs. non-adaptive
   \end{itemize}
\end{itemize}

\subsection{Adaptive sampling}
\begin{itemize}
 \item Spatio-temporal phenomena, dynamic processes
 \item What has to be done to account for environmental model changing over time?
 \item The usefulness of a measurement depends on the time at which it is taken
 \item \citet{Singh2010} incorporate temporal aspects into GP covariance functions and discuss a greedy IPP 
strategy
\end{itemize}

\unsure{Should I decide where to do further research}
\subsection{Operational issues?}
Localization, control issues, systems, collision avoidance \unsure{related to Flourish WP?}.

\subsection{Cooperative monitoring?}
Consider problems with multiple agents.
