\section{Related Work}
\label{S:related_work}

This section reviews related work in the field of informative path planning (IPP), which is the focus of 
this thesis. Generally, IPP solvers are composed of three main ingredients: 

\begin{itemize}
 \item Environmental model
 \item Utility function
 \item Planning algorithm
\end{itemize}

The developed approach aims to examine and integrate these aspects in an scalable, modular framework 
suitable for mapping and exploration. The sub-sections below introduce the basic problem before covering 
existing methods of approaching each of the three components.

\subsection{Informative path planning}

In autonomous data collection with mobile robots, a key challenge is planning paths to track evolving 
processes while respecting sensing and mobility constraints. This problem is known as informative 
path planning (IPP)~\cite{Singh2009} and has been studied extensively in the context of robotics and 
related fields. Unlike in distance-based planning, where the goal is typically to find a shortest 
route between two locations~\cite{Dijkstra1959}, the paths in IPP are subject to a budget constraint 
which limits the number of informative measurements that can be taken. The measures used to quantify 
utility are often submodular~\cite{Krause2011}, meaning that the amount of information gathered in the 
future is dependent on the prior trajectory. Hence, IPP can be viewed as procuring active 
sensing policies to close the loop between perception and planning.

Closely related to active mapping are the problems of active localization~\cite{Thrun1999} and simultaneous 
localization and mapping (SLAM)~\cite{Mu2015, Indelman2014, Bry2011}, which account for the robot's 
state uncertainty. In general, these information-gathering tasks can be formulated as partially observable 
Markov decision processes (POMDPs)~\cite{Kaelbling1998}, providing a general framework for uncertain 
planning. However, the complexity of solving high-dimensional POMDP models motivates more 
efficient solutions based on IPP. Whereas the main focus of this thesis will be active mapping with a 
fully observable robot state, later development stages will examine the possibility of integrating 
elements of belief space planning~\cite{Indelman2014, Indelman2015a} to relax this assumption.

A simplification of IPP, the NP-hard sensor placement task~\cite{Krause2008} addresses finding most 
informative 
measurement sites in a static setting. Model-aided IPP builds upon this problem by considering a sequence 
of informed decisions a mobile agent must make to retrieve an accurate model of the environment of interest. 
In its core, an IPP scheme thus requires a mapping framework to capture sensor measurements compactly, an 
objective function to quantify their utility based on the map, and a planning algorithm to generate feasible 
paths combining the map and objective.

\subsection{Environmental model}

In IPP, a model of the target environment is necessary to predict the utility of future 
measurements and capture its transversability. This model can be built from previous measurements and 
\textit{a priori} knowledge, if available, and updated incrementally by registering sensor data. Typically, 
the map representation depends on the dimensionality of the environment as well as the nature of measurements 
obtained. As in most other approaches, the proposed framework focuses on a probabilistic class of models to 
account for sensor uncertainties.

Occupancy grids are the most commonly-used type of map representation for spatial sensing in 
2D with uncorrelated measurements~\cite{Elfes1989}. This model has been distinguished for its 
efficiency and suitability for distance-based planning with search-based methods such as A* and 
D*-Lite~\cite{Koenig2002}. ~\citet{Bourgault2002} pioneered its use for active SLAM by defining entropy as 
an informative measure over the map, which has been extended in recent works~\cite{Carrillo2015a}.

The OctoMap representation~\cite{Hornung2013} is an efficient extension of volumetric occupancy 
mapping to 3D spaces. This framework uses an octree-based model to estimate the occupancy probabilities of 
voxels. The probabilities are updated dynamically from vision-based sensors – for instance, stereo 
camera imagery~\cite{Isler2016} – to yield a scalable approach to surface reconstruction while 
distinguishing unknown areas for exploration. Several recent 
works~\cite{Heng2015,Bircher2016,Charrow2015a,Isler2016} have considered this IPP set-up for single- 
and multi-object volumetric reconstruction and inspection. \citet{Nelson2015} present an alternative 
compression framework to achieve the same goal, while~\citet{Oleynikova2016} discuss the 
benefits of encoding gradient information for local planning. The proposed thesis will pursue these research 
directions by considering efficient 3D representations for planning both informative and dynamically 
feasible paths.

Often, information-gathering requires capturing complex processes with multiple influences. This 
is usually the case in environmental monitoring, where natural variables, such as weather, 
wind, and temperature, may be highly correlated. A popular probabilistic and non-parametric method of 
handling such relationships is using Gaussian Processes (GPs)~\cite{Rasmussen2006, Hitz2015, Binney2013, 
Krause2011, Singh2009, Singh2010, Vivaldini2016}. By applying regression in GP models, it is possible to 
infer measurements at unobserved locations as a basis for IPP along with their associated uncertainties. The 
field correlation and noise properties are encoded in a kernel function, whose hyperparameters can be trained 
through supervised learning techniques~\cite{Rasmussen2006}. \citet{Singh2010} analyze several kernels 
for modeling environmental processes. Building upon these ideas, GP regression outputs have been recently 
used for IPP in agriculture~\cite{Vivaldini2016}, marine biology~\cite{Hitz2015, 
Hitz2014}, and gas monitoring~\cite{Marchant2014}, among other applications.

Most environmental studies, however, do not cater for dynamic feasibility with respect to the map, and only 
consider 2D surfaces. In cluttered environments,~\citet{OCallaghan2012} extend the GP approach to construct 
3D occupancy maps, which are used subsequently by~\citet{Yang2014} for active exploration. Open problems in 
GP-based planning, however, remain model scalability and the integration of dense data from multiple sensors 
for planning in more complex environments.

Finally, some recent works have presented novel approaches to modeling, such as the topic 
summarization technique of~\citet{Girdhar2015}. Although this thesis will focus on extending traditional 
methods, it is still interesting to consider alternatives as a basis for comparison.

\subsection{Utility function}
Information-gain based mapping strategies seek to optimize a utility function reflecting the mission 
aim and constraints. Typically, this optimization problem contains two key elements: an 
information-theoretic objective and a mobility cost.

The information-theoretic objective component is used to compute the value of the sensor inputs 
obtained. Most informative measures for exploration in prior work are derived from Shannon's 
entropy~\cite{Cover2006, Lim2015, Bourgault2002, Yang2014} quantifying the map uncertainty with respect to 
its representation. To address the issue of submodularity, or diminishing returns, several methods have been 
proposed that maximize mutual information, which predicts the impact of future sensor measurements on the 
available map~\cite{Hitz2015, Hollinger2014, Krause2008}. \citet{Charrow2015a} and~\citet{Carrillo2015a} 
discuss related criteria for more efficient 3D mapping. In reconstruction from imagery, it is also common to 
simply count the number of voxels observable from a given camera position, weighting them 
accordingly~\cite{Bircher2016, Isler2016, Heng2015}. These measures encode pure exploratory aims, with the 
advantage of being readily available from the environmental models discussed above.

It may also be desirable to tailor the objective function to a specific application for interest-based 
planning. Several environmental monitoring works \cite{Gotovos2013, Hitz2014, Vivaldini2016} address 
problems of active classification, where the aim is to concentrate on areas above or below target thresholds 
within a scalar field. A related task may be to only focus on areas where sensor measurements are at a 
maximum~\cite{Marchant2014}.

The mobility component represents the physical limitations on the robot and its sensing devices. In uncertain 
environments, these give rise to the exploration-exploitation trade-off: the 
robot must choose whether to improve the map in its local vicinity, or try to visit other parts further away. 
The cost, usually expressed in terms of time, distance or fuel, can be directly encoded in the utility 
function~\cite{Charrow2015a, Bircher2016, Heng2015} or imposed as a hard budget constraint during path 
generation~\cite{Hitz2015, Binney2013}. Methods belonging to the former class usually involve a heuristically 
tuned parameter controlling the trade-off at hand; the cost element of \citet{Heng2015}, for example, 
compromises between rapid exploration and detail-filling for volumetric reconstruction.

The proposed thesis will contribute to the state-of-the-art by examining and evaluating various 
utility formulations for active mapping in a modular framework.

\subsection{Planning algorithm}
In this work, the planning algorithm of IPP is treated as the unit connecting measurement sites to form a 
feasible path in the mobile robot setting. In realistic application, the feasibility check often requires 
that the generated path does not exceed a maximal length, obeys platform-specific constraints (eg. 
quadrotor dynamics~\cite{Richter2013}), and performs safe, collision-free maneuvers. Formally, this can be 
cast as the NP-hard Orienteering Problem (OP), a combination of the well-known Knapsack Problem and 
Travelling Salesman Problem, which arises in numerous challenging applications~\cite{Vansteenwegen2011}. 
Whereas approaches from the computer science community~\cite{Vansteenwegen2011, Chekuri2005} are geared 
towards obtaining paths with theoretical optimality guarantees, IPP literature in robotics is largely 
concerned with finding practical solutions to the OP with limited computational resources.

As presented by~\citet{Lim2015}, there are two general categories of IPP algorithms in robotics: (i) 
non-adaptive and (ii) adaptive. In non-adaptive planning, a sequence of sensing operations is computed in an 
offline manner and then executed. In adaptive planning, new sensing operations are conditioned on the 
outcomes of earlier ones, allowing paths to change online as new information is incorporated. The methods 
proposed in this thesis will target the latter class to allow for safe, efficient navigation in unknown 
environments with real-time considerations.

The simplest approach to adaptive planning for active mapping is to rely on greedy one-step look-ahead 
goals. \citet{Singh2010} use such a strategy to monitor spatio-temporal temperature distributions by setting 
the next most informative predicted measurement as the target. For mapping cluttered environments, analogous 
methods can be applied based on geometric frontiers~\cite{Yamauchi1997, Latombe2002, Bircher2016, Heng2014} 
or visual targets~\cite{Kim2014}, whereas camera positions~\cite{Isler2016} are relevant candidates for 
single-object scenes. In volumetric reconstruction, greedy viewpoint selection is also known as next-best 
view planning~\cite{Connolly1985}. The main issue with this approach is that it is myopic: it does 
not perform long-term reasoning and thus tends to get stuck in local minima~\cite{Lim2015, Charrow2015a}.

Finite-horizon look-ahead searches~\cite{Lim2015, Hollinger2009, How2010} alleviate this weakness by 
extending the measurement forecast. However, their efficiency depends heavily on the predictive heuristic and 
the length of horizon used. \citet{Hitz2015} address this problem by making local modifications to a prior 
global path. Alternatively, several recent methods~\cite{Kuntz2016, Charrow2015a, Heng2015} propose a 
two-stage planning approach by combining configuration-based global exploration with local optimization to 
procure the best aspects of both paradigms. In this set-up, \citet{Heng2015} use two distinct objective 
functions accounting for efficient exploration and coverage of cluttered environments.

Finally, a common general classification for path planners is based on their operation in (i) discrete or 
(ii) continuous space. Discrete IPP algorithms~\cite{Chekuri2005, Hollinger2014, Binney2013} build upon the 
static sensor problem~\cite{Krause2008} by performing combinatorial optimization over a grid. The main 
drawbacks of such representations are their poor scalability and limited resolution. A less restrictive 
method uses a library of motion primitives~\cite{Charrow2015a, Heng2015} to generate dynamically feasible 
paths. Alternatively, continuous-space planning involves leveraging sampling-based 
methods~\cite{Hollinger2014, Bircher2016, Jadidi2016} or splines~\cite{Hitz2015, Marchant2014}, without 
requiring a pre-defined graph of measurement locations. The proposed research will consider planning with a 
view towards continuous space to cater for smooth quadrotor dynamics~\cite{Richter2013}.
