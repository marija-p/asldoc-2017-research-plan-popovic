\section*{Abstract}

Recent years have seen an increase in the use of autonomous mobile robots for flexible, cost-efficient data 
collection in a wide variety of sensing applications. However, in many cases, planning informative, 
measurement-rich paths through large, complex 3D environments remains an open challenge. This thesis aims to 
tackle the issue of information-based navigation for active exploration and mapping in 
unknown environments.

The key development will be a scalable, flexible informative path planning framework for a MAV relying 
primarily on vision-based sensing. This will be done by addressing the three key components of the planning 
pipeline: the environmental model, the utility function, and the planning algorithm. For each component, new 
approaches will be presented with a view towards planning efficiency and system integration relevant 
in practical scenarios.

The proposed methods will be evaluated in two main applications as representative use-cases of active 
mapping: agricultural monitoring and indoor exploration. The developments for each application will follow a 
phase-based schedule, gradually increasing in complexity. This approach is envisioned to evolve a 
modular planning infrastructure, which will be a key step in enabling autonomous data collection and 
furthering our understanding of the physical world.